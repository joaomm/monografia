\chapter{Introdução}
\label{chap:introducao}

A indústria de software busca continuamente por melhorias nos métodos
de desenvolvimento. Do ponto de vista prático, os processos tradicionais investem tempo especificando requisitos
e planejando a arquitetura do software para que o trabalho possa ser dividido entre os
programadores, cuja função é transformar o design anteriormente concebido em código.
Diante de uma perspectiva dos métodos ágeis, é dado muita importância a entrega constante 
de valor ao cliente \citep{AgM2001}. Neste caso, se há um agente financiador ou uma comunidade de software livre
como cliente, o foco está na entrega de funcionalidades que possam ser rapidamente colocadas
no ambiente produção e receber ``feedback'' constante.

Em ambos os casos, não há como ignorar o fato de que o código-fonte da aplicação será
desenvolvido gradativamente e diferentes programadores farão alterações e extensões
continuamente. Nesse contexto, novas funcionalidades são adicionadas e falhas sanadas.

Uma situação comum durante o desenvolvimento, é um programador lidar com um trecho que
ainda não teve contato para fazer alterações. Nessa tarefa, por exemplo, (i) lida com métodos extensos e classes
com muitas linhas de código que se comunicam com diversos objetos; (ii) encontra comentários
desatualizados relacionados a uma variável com nome abreviado e sem significado, e (iii) se depara com métodos que
recebem valores booleanos e possuem estruturas encadeadas complexas com muitas operações duplicadas.

Depois de um longo período de leitura, o desenvolvedor encontra o trecho de lógica em que deve 
fazer a mudança. A alteração freqüentemente consiste na modificação de poucas linhas e o 
desenvolvedor considera seu trabalho como terminado, sem ter feito melhorias no código confuso
com o qual teve dificuldades.

O quadro exemplificado ilustra uma circustância com aspectos a serem observados.
Primeiramente, há uma diferença significativa na proporção entre linhas lidas e as inseridas.
Para cada uma das poucas linhas que escreveu, o desenvolvedor teve que compreender diversas
outras \citep{Beck2007}. Além disso, não houve melhorias na clareza ou flexibilidade do código, fazendo 
com que os responsáveis pela próxima alteração, seja o próprio desenvolvedor ou o seguinte,
tenham que enfrentar as mesmas dificuldades.

Tendo essa situação em vista, nesse trabalho apresentamos um estilo de programação baseado
no paradigma da Orientação a Objetos que busca o que denominamos ``Código Limpo'', concebido e aplicado por renomados desenvolvedores de software
como Robert C. Martin \citep{Martin2008} e Kent Beck \citep{Beck2007}. Fundamentado em testes e decisões técnicas que visam a clareza, flexibilidade
e simplicidade do código-fonte, um código limpo pode ter um impacto importante no desenvolvimento 
de software, aumentando a produtividade e diminuindo os custos de manutenção.

Ao longo desta monografia também apresentaremos uma maneira de utilizar métricas de código-fonte para auxiliar no desenvolvimento
de um código limpo. Se beneficiando da automação da coleta de métricas e de uma maneira objetiva de interpretar seus valores,
os desenvolvedores poderão acompanhar seu trabalho e detectar cenários problemáticos.

\section{Objetivos}
\label{sec:objetivo}

O primeiro objetivo deste trabalho é apresentar um levantamento teórico de conceitos relacionados ao que denominamos
código limpo, buscando expor um conjunto de técnicas e boas decisões que possam ser adotadas ao longo do 
desenvolvimento para auxiliar a criação de um código mais expressivo, simples e flexível. Essa apresentação permite um 
contato com recomendações e preocupações de desenvolvedores de software reconhecidos internacionalmente ao trabalharem com 
o paradigma da Orientação a Objetos.

Além disso, também temos como objetivo o mapeamento entre um grupo de métricas de código-fonte e os conceitos acima 
citados, de forma a facilitar a detecção de trechos de código que poderiam receber melhorias. Para que as 
métricas possam ser mais facilmente incorporadas no cotidiano dos programadores, também temos como objetivo a apresentação 
de uma maneira de interpretar os valores das métricas através de cenários problemáticos que ocorrem frequentemente durante
o desenvolvimento.

Deste modo, queremos apresentar uma maneira de utilizar o poder da automatização das métricas associado a interpretação dos valores calculados, 
facilitando assim a aproximação para com os conceitos relacionados com código limpo.

\section{Desenvolvimento}
\label{sec:desenvolvimento}

Para definir um código limpo e selecionar o conjunto de conceitos e técnicas desta monografia, trabalhamos sobre duas principais referências: 
``Implementation Patterns'' \citep{Beck2007} de Kent Beck e ``Clean Code'' \citep{Martin2008} de Robert C. Martin. Ambos os livros foram escolhidos 
pelo grande reconhecimento dado a esses autores pela comunidade de desenvolvimento de software. Baseado em suas experiência, Beck e Martin 
expõe muitas recomendações para a criação de um bom código orientado a objetos.

Ao longo desse trabalho, escolhemos uma parcela dos conceitos comuns em ambos os livros. Como diversos aspectos tratados são bastante 
conhecidos, encontramos exemplos práticos que possam explicitar como as técnicas alteram a clareza e legibilidade do código sem se ater 
somente ao ponto de vista teórico. Por fim, trabalhamos sobre uma implementação do algoritmo de Dijkstra para encontrar a árvore geradora de 
custo mínimos, buscando encontrar melhorias que ilustrassem a aplicação dos tópicos estudados.

Para elaborar o mapeamento das métricas de código-fonte, utilizamos grande parte do conhecimento adquirido ao longo do desenvolvimento da 
ferramenta Analizo \cite{analizo2010} \cite{analizo_homepage}, capaz de extrair uma grande variedade de métricas nas linguagens C++, C e Java.
Além disso, trabalhamos sobre o livro ``Object Oriented Metrics In Practice'' \cite{Lanza06} de Michele Lanza e Radu Marinescu, que apresenta uma 
maneira bastante sistêmica de avaliar o design de aplicações orientadas a objetos. 

A união destas referências tornou possível a seleção das métricas trabalhadas. No entanto, desde o desenvolvimento e estudo sobre a Analizo, nosso grupo 
de pesquisa enfrentou dificuldades com a compreensão dos valores das métricas, o que nos motivou a focar no entendimento dos mesmos. Para tal, 
trabalhamos sobre pequenos exemplos de código utilizando o seguinte ciclo: (a) cálculo das métricas; (b) análise do código para a encontrar uma melhoria 
que pudesse aproximá-lo de um código limpo e observação dos valores das métricas para encontrar relações com a melhoria; (c) implementação da melhoria; 
(d) recálculo das métricas e análise das alterações.

Uma vez com a seleção das métricas e um bom entendimento quanto a um código limpo, trabalhamos sobre o código da ferramenta Analizo para implementar 
melhorias e tornar seu código mais limpo. Desta forma, validamos os conceitos utilizados ao longo da monografia para contribuir com o desenvolvimento de 
um software livre.

%% ------------------------------------------------------------------------- %%
\section{Organização do Trabalho}
\label{sec:organizacao_trabalho}

Além dessa introdução, este texto está organizado em quatro capítulos.

No Capítulo~\ref{chap:codigo_limpo}, apresentamos o que é um código limpo através
da união das concepções de importantes especialistas no desenvolvimento de software.
O foco será nos livros \textit{Clean Code} de Robert Martin [REF] e \textit{Implementation
Patterns} de Kent Beck [REF] e serão apresentados tantos conceitos teóricos
quanto técnicas conhecidas para a criação de um código expressivo, simples e flexível.

O capítulo seguinte, Capítulo~\ref{chap:mapeamento}, mostra o conceito das métricas de 
código-fonte e como podemos utilizá-las para extrair características importantes do
código. Além disso, apresentaremos o mapeamento objetivo entre os aspectos levantados
no capítulo anterior e as métricas para auxiliar os programadores.

No Capítulo~\ref{chap:estudodecaso} apresentamos o estudo de caso feito sobre a ferramenta Analizo para 
validar os conceitos apresentados ao longo do trabalho através da implementação de melhorias que pudessem 
tornar seu código limpo.

Por fim, as conclusões e possibilidades de trabalhos futuros serão expostos no Capítulo~\ref{chap:conclusao}.