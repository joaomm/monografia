\chapter{Estudo de Caso}
\label{chap:estudodecaso}


Utilizando as técnicas e conceitos relacionados a um código limpo e seu mapeamento 
em métricas de código fonte, apresentamos ao longo desse capítulo um estudo de caso
sobre o código da ferramenta Analizo. O objetivo é analisar sua versão atual
olhando para as métricas e as interpretações anteriormente apresentadas para detectar
melhorias possíveis. Ao final, destacaremos as principais mudanças e contribuições para
a sua ``limpeza''.


\section{Analizo}
A \textit{Analizo} \citep{analizo2010} é um conjunto de ferramentas livres para análise e visualização
do código-fonte. Ela suporta a extração e cálculo de um número significativo de métricas
de código-fonte, além da geração de grafos de dependência e o acompanhamento da evolução do software.
Seu diferencial em relação aos demais aplicativos do mesmo domínio é o suporte multi-linguagem,
o que a permite trabalhar sobre código Java, C++ e C.

A escolha da Analizo para o estudo de caso se deu devido a familiaridade com as ferramentas, 
uma vez que grande parte de seu desenvolvimento foi feito durante o projeto de Iniciação
Científica em colaboração com os doutorandos Paulo Meirelles da Universidade de São Paulo
e Antônio Terceiro da Universidade Federal da Bahia.

Durante o processo, o esforço foi dirigido na implementação de métricas para o projeto
de forma a possibilitar as pesquisas relacionadas a qualidade de 
projetos de software livre através do cálculo de métricas de código-fonte \citep{meirelles:sbqs09}, \citep{carlos_morais_2009}. Essa
realidade fez com que muito código fosse adicionado sem significativa preocupação quanto
à limpeza do código. Assim sendo, esse estudo de caso se baseia na possibilidade de promover
melhorias para o código de forma a deixar seu código mais limpo.

O foco do estudo é a ferramenta \textit{Analizo Metrics} responsável pelo cálculo de
uma quantidade significativa de métricas de código-fonte como as apresentadas anteriormente
nesse trabalho, LCOM4 e NOM, e outras como Número de Atributos Públicos e o Acoplamento
entre Objetos (CBO - \textit{Coupling Between Objects} [REF]).

Essa ferramenta é inteiramente escrita na linguagem \textit{Perl}, seguindo o paradigma da
Orientação a Objetos. Como não há aplicativo capaz de calcular as métricas selecionadas
sobre essa linguagem, todos os valores foram obtidos manualmente ao longo do desenvolvimento.

A arquitetura da \textit{Analizo Metrics} é bastante simples e composta por poucas classes.
No momento em que o usuário fornecesse um arquivo sobre o qual deseja obter as métricas, um aplicativo
auxiliar é chamada para extrair as características do código-fonte e construir um objeto da
classe \textit{Model}. Esse objeto abstrai uma representação simplificada do código e detém 
informações como usa lista de todos os módulos do código avaliado, além de características
de suas variáveis e funções \footnote{Como a ferramenta também analisa códigos não orientados
a objetos foram escolhidos os termos ``módulo'', ``variáveis'' e ``funções'' para generalizar os paradigmas.}.

O passo seguinte é feito pela classe \textit{Metrics}, que efetua o cálculo de métricas de dois 
tipos: (i) métricas de módulo como o número de linhas de funções e número de variáveis de
um dado módulo; (ii) métricas globais que contabilizam os valores do primeiro conjunto para o
cálculo de outras métricas como COF [REF] e valores estatísticos como a variância e média.

Esse estudo de caso é centrado na melhoria da classe \textit{Metrics}, cujo estado inicial continha:
(i) um conjunto de métodos que, dado um módulo, calcula o valor de uma métrica deste módulo;
(ii) um método \textit{report\_module} que retorna um dicionário que associa o nome de uma métrica
e seu valor sobre um dado módulo; e (iii) um método \textit{report} que combina os valores das métricas
de cada módulo, calcula as métricas globais e retorna um YAML contendo todos os valores.

\section{Estado Inicial}
Inicialmente, a característica central da classe \textit{Metrics} era a complexidade de seus métodos.
Eram 25 métodos que acumulavam uma alta média de LOC (aproximadamente 10 linhas de código por método),
além de uma alta média de CYCLO (3.5 por método), o que mostra que os métodos continham muitos laços condicionais.

Apesar de acumular muitas responsabilidades e ter baixa coesão, seu LCOM4 tinha um valor baixo
causado pelo método \textit{list\_of\_metrics} que não utilizava variáveis de instância e não chamava
outros métodos dentro da classe. O valor de LCOM4 igual a 2, só não era muito maior devido a presença
do método \textit{report\_module}, responsável pela chamada de todos os métodos relativos ao cálculo de
uma métrica de módulo.

Dois cenários poderiam ser observados com clareza:

\begin{itemize}
\item 
\textbf{Método Grande}

O método \textit{report} continha muitas tarefas: (i) coletar as métricas de cada um dos módulos
através de diversas chamadas para o método \textit{report\_module} ; (ii) combinar os valores em 
diversas estruturas de dados; (iii) chamar os métodos para cálculo de métricas globais; (iv)
calcular os valores estatísticos; e (v) montar a com todas as informações coletadas.

Seu corpo continha um grande número de linhas (LOC = 73) e complexidade ciclomática (CYCLO = 11).
É interessante notar que possuia um número baixo de estruturas encadeadas (MAXNESTING = 1), mostrando
que seu problema central não era a complexidade de suas tarefas, mas sua quantidade.

Esse método é um exemplo claro de método em que os detalhes de implementação ficam expostos, o que 
faz com que o leitor tenha que assimilar muitas informações para fazer uma única alteração.

\item
\textbf{Métodos com Muitos Fluxos Condicionais}

Métodos como \textit{acc} (mostrado abaixo em \ref{acc}), \textit{lcom4} e \textit{cbo}
possuiam um código relativamente pequeno, apesar de acumularem muitas estruturas encadeadas. 
Esse fato pode ser constatado através das métricas. O LOC médio dos três exemplos se aproximava de 8, 
enquanto que o MAXNESTING chegava a 5 e complexidade ciclomática 6, o que mostra que
praticamente todas as suas linhas tinham algum condicional.

O resultado desse cenário eram métodos bastante difíceis de serem compreendidos. 
Um esforço significativo teria que ser feito pelo leitor para que  pudesse afirmar
com segurança como era realizado o cálculo da métrica representado por um destes
métodos.

\lstinputlisting[label=acc, caption={Método acc com muitos fluxos condicionais}]{codigos/analizo/acc}

\end{itemize}

Um fato interessante de ser comentado é que os métodos da classe \textit{Metrics}
recebiam poucos parâmetros. Os métodos responsáveis pelo cálculo de métricas de módulo
só recebiam o módulo como parâmetro, enquanto que os demais praticamente não recebiam.

\vskip 1.0cm
\textbf{O que foi feito?}

Diante deste contexto, o primeiro passo foi utilizar a Composição de Métodos para
diminuir o tamanho dos métodos e Evitar Estruturas Encadeadas.

\section{Segundo Estado}

A tentativa de melhoria do código dos métodos da classe \textit{Metrics}
trouxeram diversas alterações e resultados que melhoraram sua expressividade.
Os métodos que inicialmente tinham muitas linhas de código foram reduzidos de forma
que a média de LOC da classe abaixasse para 5.62. Além disso, o grande número de condicionais foi
significamente reduzido, passando de uma CYCLO média de 3.5 para um valor próximo de 1.51. 

Outra contribuição importante foi a queda nos valores de MAXNESTING dos métodos
que antes se encaixavam no cenário ``Métodos com Muitos Fluxos Condionais''. A média de
MAXNESTING chegou a aproximadamente 0.53.

Essas alterações se deram devido ao grande aumento no número de métodos da classe:
os 25 métodos foram decompostos em um total de 53 novos. Considerando os valores
das métricas, podemos afirmar que os métodos do segundo estado do estudo de caso
ficaram pequenos e com aproximadamente apenas um fluxo condicional.

Pensando nas consequências para a limpeza do código, todos os métodos passaram
a ser compostos basicamente por chamadas para outros métodos com nomes explicativos
o suficiente para que a leitura de sua implementação se tornasse facilmente compreendida.
O exempo abaixo apresenta o código do método \textit{acc} que ilustra bem essa afirmação.

\lstinputlisting[label=acc2, caption={Simplicidade do método acc}]{codigos/analizo/acc2}

Com um código com essa simplicidade e expressividade, o leitor pode facilmente
afirmar que para calcular a quantidade de conexões aferentes de um dado módulo, basta
somar o número de módulos que o chamam e o número de módulos em sua árvore de herança.

Além disso, a decomposição dos métodos também trouxe melhorias quanto a flexibilidade
do código. A medida que os métodos se tornam menores, mais fácil fica a detecção de
trechos semelhantes. O encapsulamento de operações nessas circustâncias torna possível
o reúso, além de deixar os métodos clientes mais simétricos e sem muitos detalhes de implementação.

\lstinputlisting[label=parent, caption={Método que encapsula detalhes e pôde ser reutilizado}]{codigos/analizo/parent}

Em contraposição, dois cenários relacionados foram observados:

\begin{itemize}
\item
\textbf{Métodos com Muitos Parâmetros}

Com o grande aumento no número de métodos e as sucessivas chamadas que fazem entre si,
podemos notar um significativo aumento no número de parâmetros de alguns métodos.
O método \textit{add\_edges\_to\_used\_functions\_and\_variables} é um exemplo que acumulou
4 parâmetros.

\lstinputlisting[label=edges, caption={Método que acumulou muitos parâmetros}]{codigos/analizo/edges}

\item
\textbf{Muitos Repasses de Parâmetros pela Classe}
Associado ao cenário acima, podemos observar que o repasse de parâmetros se mostrou
ao longo de toda a classe \textit{Metrics}, o que deixa o corpo dos métodos poluído
com variáveis que não utilizam e somente repassam. O valor médio de parâmetros
passou de 0.64 para 1.38 e o número total de repasses foi aumentou de apenas 3 para 21.

\lstinputlisting[label=collect, caption={Método poluído pelo aumento número de parâmetros}]{codigos/analizo/collect}
\end{itemize}

É importante destacar que para implementar a Orientação a Objetos em Perl é sempre necessário
passar uma referência para o ``self''. Isso faz como que o número de parâmetros sempre
seja uma unidade maior quando comparado com as demais linguagens. Para o cálculo das métricas
esse parâmetro nunca foi contabilizado.

\vskip 1.0cm
\textbf{O que foi feito?}

O foco das próximas mudanças foi o Uso de Parâmetros como Variável de Instância para diminuir
o número de parâmetros sendo repassado.

\section{Terceiro Estado}

Como discutido na apresentação da técnica do Uso de Parâmetros como Variável de
Instância (\ref{metodos:parametros}), não podemos transformar qualquer parâmetro em 
um elemento do estado do objeto. Sendo assim, nas alterações dessa etapa não foram feitas
nenhuma alteração quanto aos parâmetros dos métodos responsáveis pelo cálculo de métricas
de módulo, uma vez que não faziam sentido para toda a classe.

No entanto, mesmo através de alterações em somente uma porção dos métodos, as
mudanças nos valores das métricas ficaram visíveis. A média de parâmetros passou
de 1.38 para 0.83, valor bastante mais próximo do inicial (0.65).

No código \ref{collect2} abaixo, podemos notar a diferença na limpeza de um método
beneficiado pela melhoria mencionada e cuja versão anterior é apresentada em \ref{collect}.

\lstinputlisting[label=collect2, caption={Método sem longas listas de parâmetros}]{codigos/analizo/collect2}

Novamente, as alterações tiveram consequências para a classe e um novo cenário pode
ser observado:

\begin{itemize}
\item
\textbf{Classe Pouco Coesa}

Ao longo da decomposição dos métodos e a promocão de parâmetros a variáveis de instância,
gradativamente podemos notar com mais clareza as responsabilidades da classe no nível da
implementação. Em outras palavras, a medida que alteramos o código, as responsabilidades
dentre uma classe começam a se confirmar pela maneira em que as tarefas são implementadas.

Desde o início desse estudo de caso, mencionamos que a classe \textit{Metrics} tinha, por exemplo, a
responsabilidade de calcular valores de métricas de módulo e valores estatísticos sobre os mesmos.
Nesse estado, podemos notar como essas responsabilidade são implementadas e o limiar entre elas,
sobre o qual trabalharemos para quebrar a classe em duas.

Essa observação conceitual se mostra através das métricas. A média de atributos atingíveis
pelos métodos passou a ser 1.1, valor significativamente menor do que o número de atributos
da classe (NOA = 6). Essa diferença mostra que existem conjuntos de métodos somente interessados
em alguns dos parâmetros.

Observando o código podemos notar que os métodos calculadores de métricas de módulo só utilizam
o objeto \textit{Model}, enquanto que os métodos relativos a coleta dos valores globais utilizam
todos os outros atributos.

\end{itemize}


\vskip 0.5cm
\textbf{O que foi feito?}

Nessa etapa muitas alterações foram feitas. A intenção foi quebrar a classe \textit{Metrics}
em várias classes com uma responsabilidade e, para cada uma delas, aplicar as mudanças apresentadas
nos outros estados.


\section{Estado Final}

Na transição para o estado final, a principal alteração foi a quebra da classe \textit{Metrics}
em muitas outras. O primeiro passo foi a criação de uma classe para uma das métricas de módulo.
Essas classes recebem o objeto \textit{Model} como parâmetro de seu construtor e possuem
um método \textit{calculate} que devolve o calor da métrica de um dado módulo.

Uma vez com essas classes isoladas, ficou mais simples detectar os parâmetros que poderiam
ser promovidos a variável de instância. Por exemplo, a nova classe \textit{LackOfCohesionOfMethods}, 
responsável pelo cálculo de LCOM4, passou a ter o atribute \textit{graph}, uma vez que 
o objeto do classe \textit{Graph} faz parte de seu estado e ciclo de vida.

O passo seguinte consistiu na identicação das demais responsabilidades da classe \textit{Metrics} 
e a criação de uma classe para cada uma delas. A classe \textit{ModuleMetrics} ficou responsável
por instânciar e invocar os objetos citados acima, enquanto que \textit{GlobalMetrics} encapsulou
a coleta das métricas globais e valores estatísticos. Essas alterações fizeram com que a classe
\textit{Metrics} inicial ficasse apenas com a responsabilidade de combinar todos os valores
calculados pelas demais classes.

\subsection{Resultados finais}
Ao final de todas as alterações, podemos notar que, de maneira geral, o código se aproximou
consideravelmente de um código expressivo, simples e flexível. O tamanho dos métodos em termos
do número de linhas e complexidade foram grandes contribuições para essa realidade. A média de
LOC passou de aproximadamente 10 para 5 e a média de CYCLO e MAXNESTING passaram, respectivamente,
de 3 para 1 e 1.2 para 0.3.

Novamente ressaltamos que esses valores foram calculados sobre uma linguagem com
características específicas. Em Perl, os parâmetros de um métodos são coletados dentro
do corpo do método, o que mostra que a média de 5 linhas seja de fato 4, uma vez que a
primeira linhas sempre é didicada a extração de pelo menos um parâmetro (```self'').

Outro aspecto que podemos observar ao final das mudanças é que de fato minimizamos
o número de parâmetros. Inicialmente, a média de número de parâmetros estava em torno de 0.65.
A medida que os métodos foram decompostos e alguns parâmetros se tornaram atributos, o valor médio
gradativamente se aproximou do valor 0.3. Essa redução ocorreu principalmente devido a criação
de classes menores e mais coesas, uma vez que seus métodos trabalham continuamente com os
atributos da mesma alterando seu estado.

Em termos das métricas, o resumo do estudo de caso é a minimização de LOC, CYCLO, MAXNESTING, NP,
LCOM4 e da diferença entre a média NRA e NOA.

Do ponto de vista conceitual, também obtivemos diversos resultados. É fácil perceber o aumento
na expressividade e simplicidade do código através dos tópicos já levantados. Para compreender
as tarefas dos métodos dentre uma classe, basta lê-lo, uma vez que sua implementação geralmente
consiste em chamadas para outros métodos cujos nomes espelham bem suas atividades. Caso a ferramenta
\textit{Analizo Metrics} fosse implementada em uma linguagem mais apropriada para a Orientação
a Objetos, os resultados poderiam ser ainda mais efetivos.

As contribuições para a flexibilidade do sistema são mais sutis. As novas classes responsáveis
pelo encapsulamento do cálculo de uma métrica de módulo associada a classe \textit{ModuleMetrics}
tem grande influência sobre esse conceito. Para adicionar uma nova métrica, basta criar uma classe
com a mesma interface das demais e adicioná-la na lista do método \textit{initialize\_calculators} de
\textit{ModuleMetrics}.

Além disso, a maximização da coesão também tem grande parte nesse papel. A medida que o código
vai sendo limpo, as responsabilidades começam a ficar mais claras. Ao quebrar as classes,
semelhanças nas abstrações ficam evidentes. Por exemplo, classes como \textit{AverageCycloComplexity}
e \textit{AverageMethodLinesOfCode}, poderiam ser refatoradas para implementar o padrão de projeto
\textit{Template Method} \citep{GOF95} através de uma herança com uma nova classe.

Um dos aspectos pouco comentados ao longo desse estudo de caso foi o acoplamento entre as classes.
Como trabalhamos sobre uma classe, poucas considerações foram feitas acerca desse conceito. No
entanto, dois tipos de interessantes puderam ser notados.

O método \textit{initialize\_calculators} da classe \textit{ModuleMetrics} usa muitos métodos
provenientes de diversas classes (invoca o método \textit{new} sobre todas), o que resulta em ECR = 1 e NCC = 15.
Claramente o método se encaixa no cenário de Método Dispersamente Acoplado.

É interessante notar que essa classe foi um resultado do esforço para criar classes coesas, tendo sido
criada para encapsular a responsabilidade de lidar com esse acomplamento. Algum método teria que fazê-lo,
então que seja dentro de uma classe e nenhum cliente saiba dessa implementação. Dessa forma, podermos dizer
que a classe \textit{ModuleMetrics} é um Objeto Centralizador.

Outro método que pode ser destacado é \textit{add\_descriptive\_statistics} da classe GlobalMetrics, cujo 
alto valor de ERC e baixo NCC igual a 1 o encaixe no cenário de Método Invejoso.

\lstinputlisting[label=stats, caption={Método Invejoso}]{codigos/analizo/stats}

De fato, seria interessante que a classe Descriptive::Statistics tivesse um método que recebesse
uma lista de valores e devolve-se um hash com chave sendo um nome de um estimador estatístico e 
com valor sendo o valor do dado estimador. Dessa forma, \textit{add\_descriptive\_statistics} não precisaria
fazer tantos pedidos para um mesmo objeto.

Esse tipo de cenário é comum quando trabalhamos com classes externas, já que não podemos controlar
sua interface. No entanto, poderíamos criar uma classe que fizesse esse trabalho e encapsulasse
a dependência.

