\chapter{Estudo de Caso}
\label{chap:estudodecaso}


Utilizando as técnicas e conceitos relacionados a um código limpo e seu mapeamento 
em métricas de código fonte, apresentaremos ao longo desse capítulo um estudo de caso
sobre o código da ferramenta Analizo. O objetivo é analisar a versão atual da ferramenta
olhando para as métricas e as interpretações anteriormente apresentadas para detectar
melhorias possíveis. Ao final, destacaremos as principais mudanças e contribuições para
a sua ``limpeza''.


\section{Analizo}
A \textit{Analizo} [REF artigo] é um conjunto de ferramentas livres para análise e visualização
do código-fonte. Ela suporta a extração e cálculo de um número significativo de métricas
de código-fonte, além da geração de grafos de dependência e o acompanhamento da evolução do software.
Seu diferencial em relação as demais ferramentas para coleta de dados é o suporte multi-linguagem,
podendo trabalhar sobre código Java, C++ e C.

A escolha da Analizo para o estudo de caso se deu devido a familiaridade com as ferramentas, 
uma vez que grande parte de seu desenvolvimento foi feito durante o projeto de Iniciação
Científica em colaboração com os doutorandos Paulo Meirelles da Universidade de São Paulo
e Antonio Terceiro da Universidade Federal da Bahia.

Durante o processo, o esforço foi dirigido na implementação de métricas para o projeto
de forma a possibilitar as pesquisas relacionadas a qualidade de software de 
projetos de software livre através do cálculo de métricas de código-fonte [REF]. Essa
realidade fez com que muito código fosse adicionado sem significativa preocupação quanto
à limpeza do código. Assim sendo, esse estudo de caso se baseia na possibilidade de promover
melhorias para o código de forma a deixar seu código mais limpo.

O foco do estudo foi a ferramenta \textit{Analizo Metrics} responsável pelo cálculo de
uma quantidade significativa de métricas de código-fonte como as apresentadas anteriormente
nesse trabalho, LCOM4 e NOM, e outras como Número de Atributos Públicos e o Acoplamento
entre Objetos (CBO - \textit{Coupling Between Objects} [REF].

Essa ferramenta é inteiramente escrita na linguagem \textit{Perl} [REF] seguindo o paradigma da
Orientação a Objetos. Como não há aplicativo capaz de calcular as métricas selecionadas
sobre essa linguagem, todos os valores foram obtidos manualmente ao longo do desenvolvimento.

A arquitetura da \textit{Analizo Metrics} é bastante simples e composta por poucas classes.
Quando o usuário fornecesse um arquivo sobre o qual deseja obter as métricas, uma ferramenta
auxiliar é chamada para extrair as características do código-fonte e construir um objeto da
classe \textit{Model}. Esse objeto abstrai uma representação simplificada do código e detém 
informações como usa lista de todos os módulos do código avaliado, além de características
de suas variáveis e funções \footnote{Como a ferramenta também analisa códigos não orientados
a objetos foram escolhidos os termos ``módulo'', ``variáveis'' e ``funções'' para generalizar os paradigmas.}.

O passo seguinte é feito pela classe \textit{Metrics}, onde é feito cálculo de métricas de dois 
tipos: (i) métricas de módulo como o número de linhas de suas funções e número de variáveis de
um dado módulo; (ii) métricas globais que contabilizam os valores do primeiro conjunto para o
cálculo de outras métricas como COF [REF] e valores estatísticos como a variância e média.

Nesse estudo de caso, o foco foi na melhoria da classe \textit{Metrics}, cujo estado inicial continha:
(i) um conjunto de métodos que, dado um módulo, calcula o valor de uma métrica deste módulo;
(ii) um método \textit{report_module} que retorna um dicionário que associa o nome de uma métrica
e seu valor sobre um dado módulo; e (iii) um método \texit{report} que combina os valores das métricas
de cada módulo, calcula as métricas globais e retorna um YAML contendo todos os valores.

