\documentclass[a4paper, 11pt]{article}
\usepackage[english,portuges]{babel}
\usepackage{indentfirst}
\usepackage[utf8x]{inputenc}
\usepackage{setspace}
\usepackage[a4paper,top=2.54cm,bottom=2.0cm,left=2.0cm,right=2.2cm]{geometry} % margens

\begin{document}
\onehalfspacing 


\title{Parte Subjetiva}
\author{João Machini de Miranda}
\maketitle

\section{Desafios e Frustrações}
Na minha opinião, nosso trabalho teve dois principais desafios. O primeiro deles era conseguir entender e
aprender bem os conceitos relacionados a um código limpo, sem muita experiência programando sistemas que tivessem
que ser mantidos posteriormente. Desde o início dos estudos, tinha em mente que gostaria de aprender a programar
bem utilizando a Orientação a Objetos e, para isso, tivemos que estudar as referências, olhar muitos códigos e juntar
ideias para assimilar bem o conteúdo.

O segundo desafio foi elaborar uma maneira de utilizar métricas de código-fonte de uma maneira que para nós parecia
interessante e produtivo. Ao longo do tempo na Iniciação Científica, frequentemente me deparava com dificuldades para
compreender os valores das métricas. Até o momento não tinha real entedimento do que significava LCOM4 = 2 e o que era
uma alta complexidade ciclomática. Desta forma, ao longo deste TCC, fizemos um esforço para compreender e conceber os
cenários de métricas. Essas ideias não estavam em nenhuma referência e esperamos que essa seja a real contribuição do trabalho.

Pensando nas frustrações, tive poucas dela, pois estou bastante satisfeito com nosso trabalho. Acredito que conseguimos fazer muito
do que concebemos inicialmente e as alterações que fizemos em seu conteúdo foi, na grande maioria das vezes, uma decisão conciente
e não por falta de tempo.

Diria que uma das minhas frustrações foi não ter encontrado o livro Code Complete. Pesquisando por mais informações em sites,
frequentemente nos deparavamos com citações a esse livro, mas infelizmente não o encontramos e, obviamente, não adicionamos seu
conteúdo na monografia.

Outra frustração foi que tive que fazer muitas disciplinas ao mesmo tempo que fazíamos o TCC. Foram cinco no primeiro semestre e
quatro no segundo. Essa realidade fez com que o tempo passasse muito rápido e tive que trabalhar bastante. Tivemos que correr
em alguns momentos.

A última frustração foi que gostaria de ter desenvolvido uma ferramenta que pudesse englobar todas as ideias do TCC.
Infelizmente isso não foi possível porque tal aplicativo demandaria muitas horas de trabalho e aumentaria muito o
escopo do trabalho.

\section{Disciplinas Relevantes}

De maneira geral, acho difícil pontuar as disciplinas relevantes para o trabalho, uma vez que grande parte dos assuntos
abordados não são cobertos pela graduação (principal razão pela qual escolhi trabalhar sobre esse tema).

\begin{itemize}
\item
\textbf{Programação Orientada a Objetos}

Como o tema principal de nosso trabalho era um Código Limpo Orientado a Objetos, essa disciplina foi bem importante
para dar uma visão mais aprofundada do tema. Apesar de não serem tratados explicitamente ao longo da monografia, os
padrões de projeto são muito relacionados com a criação de um código mais simples, flexível e expressivo. A disciplina
também foi importante porque teve EPs em Smalltalk, linguagem muito elogiada e admirada pela sua expressividade.

\item
\textbf{Laboratório de Programação II}

Essa disciplina foi a primeira a introduzir a Orientação a Objetos no curso. O que foi extremamente útil
foi que aprendemos os principais conceitos do paradigma através da sua implementação em Perl, linguagem em
que a Analizo é escrita.

\item
\textbf{Laboratório de Programação Extrema}

Essa disciplina foi importante porque foi uma oportunidade de trabalhar em um projeto maior que envolvia
programação em um grupo maior de pessoas. Desde o início, nosso grupo se compremeteu a criar muito testes
e buscar sempre um código organizado e simples. 

Como sabemos, somente com muito treino e prática desenvolvemos nossas habilidades e capacidades. Essa
disciplina foi uma boa maneira de praticar.

\item
\textbf{Engenharia de Software}

Como essa disciplina também teve um projeto de maior escala, também foi uma oportunidade para
praticar alguns conceitos, principalmente porque tive um grupo bastante empenhado em criar muitos
testes.
	
\end{itemize}
	 



\section{Futuro}

\section{Agradecimentos}
Gostaria de agradecer ao doutorando Paulo Meirelles pelo tempo que colaboramos ao longo de minha graduação.
No segundo semestre de 2009, comecei minha Iniciação Científica trabalhando integralmente com o Paulo no
estudo de ferramentas para a coleta de métricas de código-fonte. No primeiro semestre de 2010, foi cliente
do nosso grupo de Laboratório de Programação Extrema, quando comecamos o processo de desenvolvimento de uma plataforma web 
chamada Mezuro para acompanhamento de projetos que utilizava a Analizo para cálculo de métricas. Por fim,
o Paulo foi nosso coorientador durante a realização do TCC.

Também gostaria de agradecer ao recente doutor Antonio Terceiro da Universidade Federal da Bahia, principal responsável
pelo desenvolvimento da Analizo, além do pós-doutorando Carlos Denner, o mestrando Carlos Morais e o graduando Vinicius Daros
que também participaram do nosso grupo de desenvolvimento e pesquisas relacionadas as métricas de código-fonte.

Obrigado ao Professor Fabio Kon por nos orientar nesse trabalho e por ter conseguido as bolsas de Iniciação Científica,
o que possibilitou os primeiros estudos na área do TCC.

Por fim, gostaria de agradecer principalmente a minha dupla de trabalho, Lucianna. Por mais difícil que seja
trabalhar em conjunto, posso dizer que gostei muito e não escolheria fazer o trabalho sozinho se tivesse que refazê-lo.
É muito bom ter alguém para debater e organizar melhor as ideias.

\end{document}
