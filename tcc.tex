\documentclass[11pt,a4paper]{book}

% ---------------------------------------------------------------------------- %
% Pacotes 
\usepackage[T1]{fontenc}
\usepackage[brazil]{babel}
\usepackage[utf8]{inputenc}
\usepackage[pdftex]{graphicx}           % usamos arquivos pdf/png como figuras
\usepackage{setspace}                   % espaçamento flexível
\usepackage{indentfirst}                % indentação do primeiro parágrafo
\usepackage{makeidx}                    % índice remissivo
\usepackage[nottoc]{tocbibind}          % acrescentamos a bibliografia/indice/conteudo no Table of Contents
\usepackage{courier}                    % usa o Adobe Courier no lugar de Computer Modern Typewriter
\usepackage{type1cm}                    % fontes realmente escaláveis
\usepackage{listings}                   % para formatar código-fonte (ex. em Java)
\usepackage{titletoc}
%\usepackage[bf,small,compact]{titlesec} % cabeçalhos dos títulos: menores e compactos
\usepackage[fixlanguage]{babelbib}
\usepackage[font=small,format=plain,labelfont=bf,up,textfont=it,up]{caption}
\usepackage[usenames,svgnames,dvipsnames]{xcolor}
\usepackage[a4paper,top=2.54cm,bottom=2.0cm,left=2.0cm,right=2.54cm]{geometry} % margens
%\usepackage[pdftex,plainpages=false,pdfpagelabels,pagebackref,colorlinks=true,citecolor=black,linkcolor=black,urlcolor=black,filecolor=black,bookmarksopen=true]{hyperref} % links em preto
\usepackage[pdftex,plainpages=false,pdfpagelabels,pagebackref,colorlinks=true,citecolor=DarkGreen,linkcolor=NavyBlue,urlcolor=DarkRed,filecolor=green,bookmarksopen=true]{hyperref} % links coloridos
\usepackage[all]{hypcap}                    % soluciona o problema com o hyperref e capitulos
\usepackage[round,sort,nonamebreak]{natbib} % citação bibliográfica textual(plainnat-ime.bst)
\usepackage{scalefnt}
\fontsize{60}{62}\usefont{OT1}{cmr}{m}{n}{\selectfont}
\usepackage{enumerate}



% ---------------------------------------------------------------------------- %
% Cabeçalhos similares ao TAOCP de Donald E. Knuth
\usepackage{fancyhdr}
\pagestyle{fancy}
\fancyhf{}
\renewcommand{\chaptermark}[1]{\markboth{\MakeUppercase{#1}}{}}
\renewcommand{\sectionmark}[1]{\markright{\MakeUppercase{#1}}{}}
\renewcommand{\headrulewidth}{0pt}

% ---------------------------------------------------------------------------- %
%\graphicspath{{./figuras/}}             % caminho das figuras (recomendável)
\frenchspacing                          % arruma o espaço: id est (i.e.) e exempli gratia (e.g.) 
\urlstyle{same}                         % URL com o mesmo estilo do texto e não mono-spaced
\makeindex                              % para o índice remissivo
\raggedbottom                           % para não permitir espaços extra no texto
\fontsize{60}{62}\usefont{OT1}{cmr}{m}{n}{\selectfont}
\cleardoublepage
\normalsize

% ---------------------------------------------------------------------------- %
% Opções de listing usados para o código fonte
% Ref: http://en.wikibooks.org/wiki/LaTeX/Packages/Listings
\lstset{ %
language=Java,                  % choose the language of the code
basicstyle=\footnotesize,       % the size of the fonts that are used for the code
showspaces=false,               % show spaces adding particular underscores
showstringspaces=false,         % underline spaces within strings
showtabs=false,                 % show tabs within strings adding particular underscores
frame=single,	                % adds a frame around the code
framerule=0.5pt,
tabsize=2,	                    % sets default tabsize to 2 spaces
captionpos=b,                   % sets the caption-position to bottom
breaklines=true,                % sets automatic line breaking
breakatwhitespace=false,        % sets if automatic breaks should only happen at whitespace
escapeinside={\%*}{*)},         % if you want to add a comment within your code
backgroundcolor=\color[rgb]{1.0,1.0,1.0}, % choose the background color.
rulecolor=\color[rgb]{0.8,0.8,0.8},
extendedchars=true,
xleftmargin=10pt,
xrightmargin=10pt,
framexleftmargin=5pt,
framexrightmargin=5pt
}



% ---------------------------------------------------------------------------- %
% Corpo do texto
\begin{document}
\frontmatter 
% cabeçalho para as páginas das seções anteriores ao capítulo 1 (frontmatter)
\fancyhead[RO]{{\footnotesize\rightmark}\hspace{2em}\thepage}
\setcounter{tocdepth}{2}
\fancyhead[LE]{\thepage\hspace{2em}\footnotesize{\leftmark}}
\fancyhead[RE,LO]{}
\fancyhead[RO]{{\footnotesize\rightmark}\hspace{2em}\thepage}

\onehalfspacing  % espaçamento

% ---------------------------------------------------------------------------- %
% Capa
% Nota: O título para as teses/dissertações do IME-USP devem caber em um 
% orifício de 10,7cm de largura x 6,0cm de altura que há na capa fornecida pela SPG.
\thispagestyle{empty}
\begin{center}
    \vspace*{2.3cm}
    \textbf{\Large{Titulo do Trabalho}}\\
    
    \vspace*{1.2cm}
    \Large{Lucianna Thomaz Almeida}

    \vspace*{0.5cm}
	\Large{João Machini de Miranda}
    
    
    \vskip 3.5cm
    Orientador: Prof. Dr. Fabio Kon\\
    Coorientador: Paulo Roberto Miranda Meirelles

    \vskip 1.5cm
    \normalsize{São Paulo, Dezembro, 2010}
\end{center}



\pagenumbering{roman}     % começamos a numerar 
% Resumo
\chapter*{Resumo}

%Ao produzir software, os desenvolvedores não podem ficar satisfeitos com um código que simplesmente faz o trabalho que deve ser feito. É preciso considerar que será necessário manter a aplicação, fazer mudanças à medida em que os requisitos se alterarem e que outros terão que usar e aprimorar o código.
%

%Diante disso, este trabalho aborda ideias e conceitos elaborados por especialistas no desenvolvimento de software orientado a objetos, buscando um maior entendimento de boas soluções, práticas e cuidados quanto ao código-fonte. Segundo Robert Martin, um programador deveria sempre fazer com que seu código seja uma composição de instruções e abstrações que possam ser facilmente entendidas, uma vez que gastamos a maior parte do tempo lendo-o para incluir funcionalidades e corrigir falhas.
%

%Os seguintes livros são base para os nossos estudos: Clean Code de Robert Martin e Implementation Patterns de Kent Beck. Todos possuem diversos aspectos que ajudam a identificar o que é um código limpo como escolhas de bons nomes, falta de duplicações, organização e simplicidade. Esta monografia conterá uma reunião de todos esses, explicitando sua relevância e aplicação.
%

%Além de conter o levantamento teórico acima citado, esse trabalho também apresentará um mapeamento entre os conceitos encontrados na literatura em uma avaliação objetiva através de métricas de código-fonte. A ideia é possibilitar aos programadores fazer uma avaliação do software usando dados que espelham o que os especialistas recomendam, tornando possível um acompanhamento do projeto.
%

%A grande vantagem de usar métricas de código-fonte é que elas podem ser bons indicativos para os critérios subjetivos, além de serem extraídas computacionalmente com ferramentas capazes de interpretar o código-fonte e calculá-las. Diante disso, apresentaremos o Projeto Mezuro que contém ferramentas com intuito de automatizar a extração dessas métricas.


% ---------------------------------------------------------------------------- %
\mainmatter

% cabeçalho para as páginas de todos os capítulos
\fancyhead[RE,LO]{\thesection}

\singlespacing              % espaçamento simples
%\onehalfspacing            % espaçamento um e meio

\input capitulos/1-introducao
\input capitulos/2-codigolimpo

% ---------------------------------------------------------------------------- %
% Bibliografia
\backmatter \singlespacing
\bibliographystyle{plainnat-ime} % citação bibliográfica textual
\bibliography{referencias} 



\end{document}
